\documentclass{article}
\title{Module Learning With Errors}
\date{2025-08-22}
\author{Joseph Wiley Yandle}
\usepackage{amsmath}
\usepackage{amsfonts}
\usepackage{amssymb}
\usepackage{mathtools}
\usepackage[symbol]{footmisc}
\usepackage{caption}

\renewcommand{\thefootnote}{\fnsymbol{footnote}}

\newcommand\Set[2]{\{\,#1\mid#2\,\}}
\newcommand\underoverset[3]{\underset{#1}{\overset{#2}{#3}}}

\usepackage{geometry}
\geometry{
  a4paper,
  total={170mm,257mm},
  left=20mm,
  top=20mm,
}

\makeatletter
\def\@maketitle{%
  \newpage
  \null
  \vskip 2em%
  \begin{center}%
  \let \footnote \thanks
    {\Huge\bfseries\@title \par}%
    \vskip 1.5em%
    {\large
      \lineskip .5em%
      \begin{tabular}[t]{c}%
        \@author \footnote{xoloki@gmail.com}
      \end{tabular}\par}%
    \vskip 1em%
    {\large \@date}%
  \end{center}%
  \par
  \vskip 1.5em}
\makeatother

\begin{document}
\onecolumn
\maketitle

\begingroup
\leftskip5em
\parskip1em
\rightskip\leftskip
\noindent\textbf{Abstract.} An exploration of the math behind M-LWE

\par
\noindent\textbf{Keywords:} Lattice, Polynomial Ring, Learning With Errors, Post Quantum Cryptography
\par
\endgroup

\section{
  Introduction
}

M-LWE \emph{(Module Learning With Errors)} \cite{mlwe} is a lattice based post-quantum cryptosystem that can be used for encryption, key exchange, and digital signatures.  It is based on LWE \cite{lwe}, and the later improvement Ring-LWE \cite{ringlwe}, but tries to balance the performance gains of the latter against the greater security of the former.

Like Ring-LWE, M-LWE uses quotient rings of polynomials rather than lattices directly.  The hardness of these ring computations reduces to that of standard lattice hard problem (CVP, SVP, SIS), but ring computations are generally faster.  Ring-LWE, though, reduces to a restricted class of lattices with less strict security guarantees, while M-LWE is more flexible.  Thus M-LWE can provide a balance of performance and security, depending on the parameters chosen.

\section{
  Background
}

\subsection{
  Lattices
}

In linear algebra, vector spaces are composed from linear combinations of basis vectors over real numbers.  Lattices are discrete vector spaces, which means that only integer combinations of the basis vectors are allowed.  In two dimensions, the resulting graph resembles a garden trellis, hence the name lattice.



\subsection{
  Rings
}

\subsubsection{
  Quotient Rings
}

\subsubsection{
  Polynomial Rings
}

\subsubsection{
  Quotient Rings of Polynomials
}

\newpage
\section{
  Module Learning with Errors
}

\subsection{
  Setup
}

M-LWE requires a set of parameters, which include the polynomial degree $n$, the module rank $d$, the modulus $q$, and the bound $\eta$.  $\eta$ will be used to construct small polynomials, which are used to provide errors and randomness which guard the data, and must be significantly smaller than $q$ to allow recovery of plaintext and verification of signatures.

Once the parameters are set, you can construct public and private keys.  First construct a rank $d$ square matrix $A$ of degree $n$ polynomials, whose coefficients are integers $\mod{q}$.  Next construct two rank $d$ vectors of degree $n$ polynomials whose coefficients are bounded by $(-\eta, +\eta)$: private key $s$ and error $e$.  Then compute $t = A \cdot s + e \pmod{q}$.  The public key is $(A, t)$.

\subsection{
  Encryption
}

To encrypt a message $m$, first encode it as a binary polynomial (all coefficients $0$ or $1$) of degree $n$.  Then construct several random rank $d$ vector of degree $n$ polynomials, whose coefficients are bounded by $(-\eta, \eta)$ (i.e. a small polynomial).  These include $r$, which provides randomness; and $e_1, e_2$ which provide errors.

Next construct $u = A^T \cdot r + e_1 \pmod{q}$, then scale binary polynomial $m$ by a factor of $q/2$ to get $m_s$.  Then you can compute the final value $v = \langle t, r \rangle +\;e_2 + m_s \pmod{q}$.  The ciphertext is $(u, v)$.

\subsection{
  Decryption
}

To decrypt ciphertext $(u, v)$, construct approximate value $m_a = v - \langle u, s \rangle \pmod{q}$.  This approximates $m_s$ since:
\begin{align}
  v - \langle u, s \rangle &= \langle t, r \rangle + e_2 + m_s - \langle u, s \rangle\nonumber\\
  &= \langle A \cdot s + e, r \rangle + e_2 + m_s - \langle A^T \cdot r + e_1, s \rangle\nonumber\\
  &\approx \langle A \cdot s, r \rangle + m_s - \langle A^T \cdot r, s \rangle\nonumber\\
  &\approx (A \cdot s)^T \cdot r + m_s - (A^T \cdot r)^T \cdot s\nonumber\\
  &\approx (A \cdot s)^T \cdot r + m_s - (A \cdot r^T) \cdot s\nonumber\\
  &\approx (A \cdot s)^T \cdot r + m_s - r^T \cdot (A \cdot s)\nonumber\\
  &\approx (A \cdot s)^T \cdot r + m_s - (A \cdot s)^T \cdot r\nonumber\\
  &\approx m_s
\end{align}

Finally, remove the scaling factor $q/2$ from $m_s$: reduce the coefficients to $1$ if they are closer to $q/2$ than $0$, else reduce to $0$.  If $q$ is sufficiently larger than $\eta$, this will recover the plaintext $m$ with high degree of probability.


\begin{thebibliography}{3}

\bibitem{lwe}
  Oded Regev
  \emph{On Lattices, Learning with Errors, Random Linear Codes, and Cryptography}
  \texttt{https://arxiv.org/pdf/2401.03703}

\bibitem{ringlwe}
  Vadim Lyubashevsky, Chris Peikert, Oded Regev
  \emph{On Ideal Lattices and Learning with Errors Over Rings}
  \texttt{https://eprint.iacr.org/2012/230.pdf}

\bibitem{mlwe}
  Adeline Langlois, Damien Stehlé
  \emph{Worst-Case to Average-Case Reductions for Module Lattices}
  \texttt{https://eprint.iacr.org/2012/090.pdf}
  
\end{thebibliography}

\end{document}

