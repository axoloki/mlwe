\documentclass{article}
\title{Module Learning With Errors}
\date{2025-08-22}
\author{Joseph Wiley Yandle}
\usepackage{amsmath}
\usepackage{amsfonts}
\usepackage{amssymb}
\usepackage{mathtools}
\usepackage[symbol]{footmisc}
\usepackage{caption}

\renewcommand{\thefootnote}{\fnsymbol{footnote}}

\newcommand\Set[2]{\{\,#1\mid#2\,\}}
\newcommand\underoverset[3]{\underset{#1}{\overset{#2}{#3}}}

\usepackage{geometry}
\geometry{
  a4paper,
  total={170mm,257mm},
  left=20mm,
  top=20mm,
}

\makeatletter
\def\@maketitle{%
  \newpage
  \null
  \vskip 2em%
  \begin{center}%
  \let \footnote \thanks
    {\Huge\bfseries\@title \par}%
    \vskip 1.5em%
    {\large
      \lineskip .5em%
      \begin{tabular}[t]{c}%
        \@author \footnote{xoloki@gmail.com}
      \end{tabular}\par}%
    \vskip 1em%
    {\large \@date}%
  \end{center}%
  \par
  \vskip 1.5em}
\makeatother

\begin{document}
\onecolumn
\maketitle

\begingroup
\leftskip5em
\parskip1em
\rightskip\leftskip
\noindent\textbf{Abstract.} An exploration of the math behind M-LWE

\par
\noindent\textbf{Keywords:} Lattice, Polynomial Ring, Learning With Errors, Post Quantum Cryptography
\par
\endgroup

\section{
  Introduction
}

M-LWE \emph{(Module Learning With Errors)} is a lattice based post-quantum cryptosystem that can be used for encryption, key exchange, and digital signatures.  It is based on LWE \cite{lwe}, and the later improvement Ring-LWE \cite{ring-lwe}.

\section{
  Setup
}

M-LWE requires a set of parameters, which include the polynomial degree $n$, the module rank $d$, the modulus $q$, and the bound $\eta$.  $\eta$ will be used to construct small polynomials, which are used to provide errors and randomness which guard the data, and must be significantly smaller than $q$ to allow recovery of plaintext and verification of signatures.

Once the parameters are set, you can construct public and private keys.  First construct a rank $d$ square matrix $A$ of degree $n$ polynomials, whose coefficients are integers $\mod{q}$.  Next construct two rank $d$ vectors of degree $n$ polynomials whose coefficients are bounded by $(-\eta, +\eta)$: private key $s$ and error $e$.  Then compute $t = A \cdot s + e \pmod{q}$.  The public key is $(A, t)$.

\newpage
\section{
  Encryption
}

To encrypt a message $m$, first encode it as a binary polynomial (all coefficients $0$ or $1$) of degree $n$.  Then construct several random rank $d$ vector of degree $n$ polynomials, whose coefficients are bounded by $(-\eta, \eta)$ (i.e. a small polynomial).  These include $r$, which provides randomness; and $e_1, e_2$ which provide errors.

Next construct $u = A^T \cdot r + e_1 \pmod{q}$, then scale binary polynomial $m$ by a factor of $q/2$ to get $m_s$.  Then you can compute the final value $v = \langle t, r \rangle +\;e_2 + m_s \pmod{q}$.  The ciphertext is $(u, v)$.

\section{
  Decryption
}

To decrypt ciphertext $(u, v)$, construct approximate value $m_a = v - \langle u, s \rangle \pmod{q}$.  This approximates $m_s$ since:
\begin{align}
  v - \langle u, s \rangle &= \langle t, r \rangle + e_2 + m_s - \langle u, s \rangle\nonumber\\
  &= \langle A \cdot s + e, r \rangle + e_2 + m_s - \langle A^T \cdot r + e_1, s \rangle\nonumber\\
  &\approx \langle A \cdot s, r \rangle + m_s - \langle A^T \cdot r, s \rangle\nonumber\\
  &\approx (A \cdot s)^T \cdot r + m_s - (A^T \cdot r)^T \cdot s\nonumber\\
  &\approx (A \cdot s)^T \cdot r + m_s - (A \cdot r^T) \cdot s\nonumber\\
  &\approx (A \cdot s)^T \cdot r + m_s - r^T \cdot (A \cdot s)\nonumber\\
  &\approx (A \cdot s)^T \cdot r + m_s - (A^T \cdot s) \cdot r\nonumber\\
  &\approx m_s
\end{align}

Finally, remove the scaling factor $q/2$ from $m_s$: reduce the coefficients to $1$ if they are closer to $q/2$ than $0$, else reduce to $0$.  If $q$ is sufficiently larger than $\eta$, this will recover the plaintext $m$ with high degree of probability.


\begin{thebibliography}{2}

\bibitem{lwe}
  Oded Regev
  \emph{On Lattices, Learning with Errors, Random Linear Codes, and Cryptography} 2024
  \texttt{https://arxiv.org/pdf/2401.03703}

\bibitem{ring-lwe}
  Vadim Lyubashevsky, Chris Peikert, Oded Regev
  \emph{On Ideal Lattices and Learning with Errors Over Rings} 2013
  \texttt{https://eprint.iacr.org/2012/230.pdf}
  
\end{thebibliography}



\end{document}

